\documentclass[twoside,a4paper]{article}
\usepackage[utf8]{inputenc}
\usepackage[a4paper,bindingoffset=0.5cm,inner=2.5cm,outer=2cm,top=2.5cm,bottom=2.5cm]{geometry}
%\usepackage[a4paper,bindingoffset=1cm,inner=2cm,outer=2cm,top=2.5cm,bottom=2.5cm]{geometry} %,showframe
\usepackage{helvet}
\usepackage[T1]{fontenc}
\renewcommand{\familydefault}{\sfdefault}
% \usepackage[german,ngerman]{babel}
\usepackage[english]{babel}
\usepackage{graphicx}
\usepackage{amsmath,amsthm,amstext,amssymb,bm}
\usepackage{xcolor,color}
\usepackage{pifont}
\usepackage{array}
\usepackage{upgreek}
\usepackage{enumerate}
\usepackage{pgf}
\usepackage{mathrsfs}
\usepackage{subfig}
\usepackage{tabularx}
\usepackage[numbers]{natbib}
\usepackage{multirow}
\usepackage{makecell}
%%% Title page
\usepackage{common/titlePageST}
%%% Appendix in TOC
\usepackage[toc,page]{appendix}
%%% Setstrech on title page
\usepackage{setspace}
%%% Tilde in URL in literature
\usepackage{url}
%%% multiple rows
\usepackage{multirow}
%%% fancy column and row seperators
\usepackage{hhline}
%%% custom items in enumerate and itemize
\usepackage{enumitem}

\clubpenalty=5000
\widowpenalty=5000

\setlength{\emergencystretch}{2cm}
%----------------------------------------------------------------------------------------
%	abbreviation includes
%----------------------------------------------------------------------------------------
%%% Standard Math
% todo function
\newcommand{\todo}[1]{\textcolor{red}{#1}}

% spaces
\newcommand{\N}{\ensuremath{\mathbb{N}}}
\newcommand{\R}{\ensuremath{\mathbb{R}}}
\newcommand{\Rpos}{\ensuremath{\mathbb{R}_{>0}}}
\newcommand{\Rnneg}{\ensuremath{\mathbb{R}_{\geq 0}}}
% Complex numbers
%		Problem: \C does not work with XeLaTeX
\newcommand{\Cn}{\ensuremath{\mathbb{C}}}
% imaginary unit
\newcommand{\imagUnit}{\ensuremath{{1i}}}

% formation
\newcommand{\txtfrac}[2]{\ensuremath{{#1}/{#2}}}

% braces
\newcommand{\lb}{\left(}
\newcommand{\rb}{\right)}
\newcommand{\lsb}{\left[}
\newcommand{\rsb}{\right]}
\newcommand{\lcb}{\left\{}
\newcommand{\rcb}{\right\}}

% Operators
\DeclareMathOperator*{\argmin}{\operatornamewithlimits{argmin}}
\DeclareMathOperator*{\argmax}{\operatornamewithlimits{argmax}}
\let\Re\relax
\DeclareMathOperator{\Re}{Re}
\newcommand{\Reb}[1]{\ensuremath{\Re\lb{#1}\rb}}
\let\Im\relax
\DeclareMathOperator{\Im}{Im}
\newcommand{\Imb}[1]{\ensuremath{\Im\lb{#1}\rb}}
\DeclareMathOperator*{\rank}{rank}
\newcommand{\rankb}[1]{\rank\lb{#1}\rb}
\DeclareMathOperator*{\img}{img}
\DeclareMathOperator*{\trace}{trace}
\DeclareMathOperator*{\mspan}{span} % Math SPAN
\DeclareMathOperator*{\colspan}{colspan}
\DeclareMathOperator*{\diag}{diag}
\DeclareMathOperator*{\blkdiag}{blkdiag}
\newcommand{\detb}[1]{\ensuremath{\det{\lb #1 \rb}}}
\newcommand{\blkdiagb}[1]{\ensuremath{\blkdiag{\lb #1 \rb}}}
\newcommand{\diagb}[1]{\ensuremath{\diag{\lb #1 \rb}}}
\newcommand{\dimb}[1]{\ensuremath{\dim{\lb #1 \rb}}}
\newcommand{\expb}[1]{\ensuremath{\exp{\lb #1 \rb}}}
\newcommand{\imgb}[1]{\ensuremath{\img{\lb #1 \rb}}}
\newcommand{\colspanb}[1]{\ensuremath{\colspan{\lb #1 \rb}}}
\newcommand{\traceb}[1]{\trace\lb{#1}\rb}
\newcommand{\spanb}[1]{\ensuremath{\mspan{\lcb #1 \rcb}}}
\newcommand{\sinb}[1]{\ensuremath{\sin{\lb #1 \rb}}}
\newcommand{\abs}[1]{\ensuremath{\left|{#1}\right|}}
\newcommand{\minimize}[1]{\underset{#1}{\operatorname{minimize}}}
\newcommand{\maximize}[1]{\underset{#1}{\operatorname{maximize}}}

%%% Fat alphabet
% Small letters
\newcommand{\fa}{\ensuremath{\bm{a}}}
\newcommand{\fb}{\ensuremath{\bm{b}}}
\newcommand{\fc}{\ensuremath{\bm{c}}}
\newcommand{\fd}{\ensuremath{\bm{d}}}
\newcommand{\fe}{\ensuremath{\bm{e}}}
\newcommand{\ff}{\ensuremath{\bm{f}}}
\newcommand{\fg}{\ensuremath{\bm{g}}}
\newcommand{\fh}{\ensuremath{\bm{h}}}
\newcommand{\fati}{\ensuremath{\bm{i}}}
\newcommand{\fj}{\ensuremath{\bm{j}}}
\newcommand{\fk}{\ensuremath{\bm{k}}}
\newcommand{\fl}{\ensuremath{\bm{l}}}
\newcommand{\fm}{\ensuremath{\bm{m}}}
\newcommand{\fn}{\ensuremath{\bm{n}}}
\newcommand{\fo}{\ensuremath{\bm{o}}}
\newcommand{\fp}{\ensuremath{\bm{p}}}
\newcommand{\fq}{\ensuremath{\bm{q}}}
\newcommand{\fr}{\ensuremath{\bm{r}}}
\newcommand{\fs}{\ensuremath{\bm{s}}}
\newcommand{\ft}{\ensuremath{\bm{t}}}
\newcommand{\fu}{\ensuremath{\bm{u}}}
\newcommand{\fv}{\ensuremath{\bm{v}}}
\newcommand{\fw}{\ensuremath{\bm{w}}}
\newcommand{\fx}{\ensuremath{\bm{x}}}
\newcommand{\fy}{\ensuremath{\bm{y}}}
\newcommand{\fz}{\ensuremath{\bm{z}}}

% Capital letters
\newcommand{\fA}{\ensuremath{\bm{A}}}
\newcommand{\fB}{\ensuremath{\bm{B}}}
\newcommand{\fC}{\ensuremath{\bm{C}}}
\newcommand{\fD}{\ensuremath{\bm{D}}}
\newcommand{\fE}{\ensuremath{\bm{E}}}
\newcommand{\fF}{\ensuremath{\bm{F}}}
\newcommand{\fG}{\ensuremath{\bm{G}}}
\newcommand{\fH}{\ensuremath{\bm{H}}}
\newcommand{\fI}{\ensuremath{\bm{I}}}
\newcommand{\fJ}{\ensuremath{\bm{J}}}
\newcommand{\fK}{\ensuremath{\bm{K}}}
\newcommand{\fL}{\ensuremath{\bm{L}}}
\newcommand{\fM}{\ensuremath{\bm{M}}}
\newcommand{\fN}{\ensuremath{\bm{N}}}
\newcommand{\fO}{\ensuremath{\bm{O}}}
\newcommand{\fP}{\ensuremath{\bm{P}}}
\newcommand{\fQ}{\ensuremath{\bm{Q}}}
\newcommand{\fR}{\ensuremath{\bm{R}}}
\newcommand{\fS}{\ensuremath{\bm{S}}}
\newcommand{\fT}{\ensuremath{\bm{T}}}
\newcommand{\fU}{\ensuremath{\bm{U}}}
\newcommand{\fV}{\ensuremath{\bm{V}}}
\newcommand{\fW}{\ensuremath{\bm{W}}}
\newcommand{\fX}{\ensuremath{\bm{X}}}
\newcommand{\fY}{\ensuremath{\bm{Y}}}
\newcommand{\fZ}{\ensuremath{\bm{Z}}}

% Greek small letters
\newcommand{\falpha}{\ensuremath{\bm{\alpha}}}
\newcommand{\fbeta}{\ensuremath{\bm{\beta}}}
\newcommand{\fgamma}{\ensuremath{\bm{\gamma}}}
\newcommand{\fdelta}{\ensuremath{\bm{\delta}}}
\newcommand{\fepsilon}{\ensuremath{\bm{\vareps}}}
\newcommand{\fzeta}{\ensuremath{\bm{\zeta}}}
\newcommand{\feta}{\ensuremath{\bm{\eta}}}
\newcommand{\ftheta}{\ensuremath{\bm{\theta}}}
\newcommand{\flota}{\ensuremath{\bm{\lota}}}
\newcommand{\fkappa}{\ensuremath{\bm{\kappa}}}
\newcommand{\flambda}{\ensuremath{\bm{\lambda}}}
\newcommand{\fmu}{\ensuremath{\bm{\mu}}}
\newcommand{\fnu}{\ensuremath{\bm{\nu}}}
\newcommand{\fxi}{\ensuremath{\bm{\xi}}}
% skipped omikron
\newcommand{\fpi}{\ensuremath{\bm{\pi}}}
\newcommand{\frho}{\ensuremath{\bm{\rho}}}
\newcommand{\fsigma}{\ensuremath{\bm{\sigma}}}
\newcommand{\ftau}{\ensuremath{\bm{\tau}}}
\newcommand{\fupsilon}{\ensuremath{\bm{\upsilon}}}
\newcommand{\fphi}{\ensuremath{\bm{\varphi}}}
\newcommand{\fchi}{\ensuremath{\bm{\chi}}}
\newcommand{\fpsi}{\ensuremath{\bm{\psi}}}
\newcommand{\fomega}{\ensuremath{\bm{\omega}}}

% some Greek capital letters
\newcommand{\fGamma}{\ensuremath{\bm{\Gamma}}}
\newcommand{\fDelta}{\ensuremath{\bm{\Delta}}}
\newcommand{\fTheta}{\ensuremath{\bm{\Theta}}}
\newcommand{\fLambda}{\ensuremath{\bm{\Lambda}}}
\newcommand{\fXi}{\ensuremath{\bm{\Xi}}}
\newcommand{\fPi}{\ensuremath{\bm{\Pi}}}
\newcommand{\fSigma}{\ensuremath{\bm{\Sigma}}}
\newcommand{\fUpsilon}{\ensuremath{\bm{\Upsilon}}}
\newcommand{\fPhi}{\ensuremath{\bm{\varPhi}}}
\newcommand{\fPsi}{\ensuremath{\bm{\Psi}}}
\newcommand{\fOmega}{\ensuremath{\bm{\varOmega}}}

% Numbers
\newcommand{\fzero}{\ensuremath{\bm{0}}}
\newcommand{\fone}{\ensuremath{\bm{1}}}

%%% Standard mathematical operators
% (real) Transpose operator
%		Problem: \T does not work with XeLaTeX
\newcommand{\rT}[1]{\ensuremath{#1^{\textsf{T}}}}
\newcommand{\rTb}[1]{\ensuremath{\rT{\lb#1\rb}}}
\newcommand{\rTsb}[1]{\ensuremath{\rT{\lsb#1\rsb}}}

% Conjugate transpose
\newcommand{\cT}[1]{\ensuremath{#1^{\ast}}}
\newcommand{\cTb}[1]{\ensuremath{\cT{\lb{}#1\rb}}}

% Hermitian operator
\newcommand{\Her}[1]{\ensuremath{{#1}^{\textsf{H}}}}
\newcommand{\Herb}[1]{\ensuremath{\Her{\lb{}#1\rb}}}

% Symplectic inverse
\newcommand{\si}[1]{\ensuremath{#1^{+}}}
\newcommand{\sib}[1]{\ensuremath{\si{\lb{}#1\rb}}}

% orthogonal complement
\newcommand{\orth}[1]{\ensuremath{#1^{\perp}}}

% Inverse operator
\newcommand{\inv}[1]{\ensuremath{{#1}^{\textsf{-1}}}}
\newcommand{\invb}[1]{\ensuremath{\inv{\lb{#1}\rb}}}

% Inverse transposed operator
\newcommand{\invT}[1]{\ensuremath{{#1}^{\textsf{-T}}}}
\newcommand{\invTb}[1]{\ensuremath{\invT{\lb{#1}\rb}}}

% Gradients
\newcommand{\grad}[1][]{\ensuremath{\nabla_{#1}}}

% Divergence
\let\div\relax
\DeclareMathOperator{\div}{div}
\newcommand{\divb}[2][]{\ensuremath{\div_{#1}\lb{#2}\rb}}

% parial derivative
\newcommand{\deldel}[2][]{\ensuremath{\frac{\partial #1}{\partial #2}}}
\newcommand{\deldelt}{\ensuremath{\deldel{t}}}
\newcommand{\deldelfx}{\ensuremath{\deldel{\fx}}}

% second-order partial derivative
\newcommand{\sodeldel}[1]{\ensuremath{\frac{\partial^2}{\partial^2 #1}}} % Second-Order DELDEL

% jacobian
\newcommand{\jac}[2]{\deldel[#1]{#2}}
\newcommand{\jacp}[2]{\lb
    \jac{#1}{#2}
\rb}

% Derivatives written out
\newcommand{\dd}[2][]{\ensuremath{\frac{\mathrm{d}^{#1}}{\mathrm{d}#2}}}
\newcommand{\ddfx}{\ensuremath{\dd{\fx}}}
\newcommand{\ddfy}{\ensuremath{\dd{\fy}}}
\newcommand{\ddt}{\ensuremath{\dd{t}}}
\newcommand{\dds}{\ensuremath{\dd{s}}}

% Second order derivative written out
\newcommand{\soddt}{\dd[2]{t^2}}

% Integral operators
\newcommand{\intd}[1]{\ensuremath{\mathrm{d}#1}} % integral d (\d already in use)
\newcommand{\ds}{\ensuremath{\intd{s}}}

% Symplectic matrix
\newcommand{\Jtwo}[1]{\ensuremath{\mathbb{J}_{2#1}}}
\newcommand{\TJtwo}[1]{\ensuremath{\rT{\mathbb{J}}_{2#1}}}
\newcommand{\Jtn}{\ensuremath{\Jtwo{n}}}
\newcommand{\TJtn}{\ensuremath{\TJtwo{n}}}
\newcommand{\Jtk}{\ensuremath{\Jtwo{k}}}
\newcommand{\TJtk}{\ensuremath{\TJtwo{k}}}
\newcommand{\J}{\ensuremath{\mathbb{J}}}

% Identity matrix
\newcommand{\I}[1]{\ensuremath{\fI_{#1}}}
\newcommand{\In}{\ensuremath{\I{n}}}

% Matrix of zeros
\newcommand{\Z}[1]{\ensuremath{\fzero}_{#1}}
\newcommand{\Zn}{\ensuremath{\Z{n}}}

% Matrix of ones
\newcommand{\Ones}[1]{\ensuremath{\fone}_{#1}}

% Norms
\newcommand{\norm}[1]{\left\lVert{#1}\right\rVert}
\newcommand{\wnorm}[2]{\ensuremath{\norm{#2}_{#1}}} % weighted norm
\newcommand{\Fnorm}[1]{\wnorm{\mathrm{F}}{#1}}
\newcommand{\tnorm}[1]{\wnorm{\mathrm{2}}{#1}} % two norm

% Inner products
\newcommand{\ip}[2]{\left\langle #1,\; #2 \right\rangle}
\newcommand{\wip}[3]{\ip{#2}{#3}_{#1}}

% Special matrices
\newcommand{\sq}[1]{\ensuremath{{#1}^{\frac{1}{2}}}}
\newcommand{\isq}[1]{\ensuremath{{#1}^{-\frac{1}{2}}}}

% standard ODE stuff
\newcommand{\tInit}{\ensuremath{t_{\mathrm{0}}}}
\newcommand{\tEnd}{\ensuremath{t_{\mathrm{end}}}}
\newcommand{\ftInterval}{\ensuremath{[\tInit, \tEnd]}} % finite time interval
\newcommand{\fxInit}{\ensuremath{\fx_{\mathrm{0}}}}

% thesis specific commands

\newcommand{\qpvec}{\begin{pmatrix}
    q \\
    p
\end{pmatrix}}

\newcommand{\uppersympop}[1]{\begin{bmatrix}
    I & {#1} \\
    0 & I
\end{bmatrix} \qpvec}

\newcommand{\lowersympop}[1]{\begin{bmatrix}
    I & 0 \\
    {#1} & I
\end{bmatrix} \qpvec}

\newcommand{\activation}{\sigma}

\newcommand{\onevec}[1]{1_{#1}}
%%% Clever refing
\usepackage[capitalise,noabbrev]{cleveref}

%----------------------------------------------------------------------------------------
%	References with cleveref
%----------------------------------------------------------------------------------------
\crefformat{equation}{(#2#1#3)}

%----------------------------------------------------------------------------------------
%	Theorem environments
%----------------------------------------------------------------------------------------
\newtheorem{theorem}{Theorem}
\newtheorem{example}{Example}
\newtheorem{corollary}{Corollary}

%----------------------------------------------------------------------------------------
%	fancyhdr
%----------------------------------------------------------------------------------------
\usepackage{fancyhdr}

\makeatletter
\newcommand{\theauthor}{Steffen Müller} %
\makeatother

\renewcommand{\headrulewidth}{0.3pt}
\renewcommand{\footrulewidth}{0.3pt}
\fancyfoot[OL,ER]{University of Stuttgart} % inner
\fancyfoot[OR,EL]{\thepage} % outer
\fancyhead[OL,ER]{\theauthor} % inner 
\fancyhead[OR,EL]{IANS -- Institute of Applied Analysis and Numerical Simulation} % outer
\fancyfoot[C]{}

% \headsep=4mm
% \footskip=4mm
\parindent=0mm
\parskip=6pt
% \renewcommand{\footskip}{3pt}

%----------------------------------------------------------------------------------------
%	Something
%----------------------------------------------------------------------------------------
\usepackage{textpos}
\setlength{\TPHorizModule}{1mm}%
\setlength{\TPVertModule}{1mm}%
% \headsep=5mm
% \footskip=5mm
\pagestyle{fancy}
\newcommand{\articleheading}[3]{
{\large #1}\\[3mm]
{\Large\bf #2}\\[3mm]
{\large #3}
}

%----------------------------------------------------------------------------------------
%	Start Document
%----------------------------------------------------------------------------------------
\begin{document}
\pagenumbering{roman}
%----------------------------------------------------------------------------------------
%
% TITLE PAGE
%
%----------------------------------------------------------------------------------------
%------------------------------------------
\begin{titlePageST}
%------------------------------------------
\makeLogo%
{-10pt}{
\includegraphics[width=0.7\textwidth]{figures/logos/simtech.pdf}
}%
{0pt}{
	\begin{center}
		\includegraphics[width=0.4\textwidth]{figures/logos/ians.pdf}
	\end{center}}%
{0pt}{
\begin{flushright}
	\vspace{-10pt}
	\includegraphics[width=0.9\textwidth]{figures/logos/unistuttgart_logo_englisch_cmyk.eps}
\end{flushright}}%
\vspace{35pt}%
%------------------------------------------
\makeHeader%
[Research Group: Numerical Mathematics] %
{Institute of Applied Analysis and Numerical Simulation} %
\vspace{50pt}%
%------------------------------------------
\makeTitle%
{Simulation Technology Degree Course} %
{Bachelor Thesis} %
\vspace{80pt}%
%------------------------------------------
\makeTitleThesis%
{Symplectic Neural Networks}
\vspace{90pt}%
%------------------------------------------
\begin{supervisorST}{3}%
\addSuper%
{First Reviewer}%
{Prof. Dr. B. Haasdonk}%
{Institute of Applied Analysis and\\[-0.2cm]
Numerical Simulation (IANS)}%
\addSuper%
{Second Reviewer}%
{Prof. Dr. D. Pflüger}%
{Institute of Parallel and Distributed\\[-0.2cm]
Systems (Scientific Computing)}%
\addSuper%
{Advisor}%
{Patrick Buchfink, M.Sc.}%
{Institute of Applied Analysis and\\[-0.2cm]
Numerical Simulation (IANS)}%
\end{supervisorST}%
\vspace{80pt}%
%------------------------------------------
\begin{authorST}{Submitted by}%
\addAuthorInfo{Author}{Steffen Müller}
\addAuthorInfo{Student ID}{3260643}
\addAuthorInfo{SimTech ID}{...} %FILLIN
\addAuthorInfo{Submission Date}{...} %FILLIN
\end{authorST}%
%------------------------------------------
\end{titlePageST}
%----------------------------------------------------------------------------------------
\clearpage
\newpage\thispagestyle{plain}\null
\newpage\thispagestyle{plain}

%----------------------------------------------------------------------------------------
%
% ABSTRACT
%
%----------------------------------------------------------------------------------------
\section*{Abstract}
...
\clearpage
\newpage\thispagestyle{plain}\null
\newpage\thispagestyle{plain}

%----------------------------------------------------------------------------------------
%
% ACKNOWLEDGEMENTS
%
%----------------------------------------------------------------------------------------
\section*{Acknowledgements}
...
\clearpage
\newpage\thispagestyle{plain}\null
\newpage\thispagestyle{plain}

%----------------------------------------------------------------------------------------
%
% TOC
%
%----------------------------------------------------------------------------------------
\setcounter{tocdepth}{2}
\tableofcontents
\clearpage
\newpage\thispagestyle{plain}\null
%----------------------------------------------------------------------------------------
%
% Begin with document content
%
%----------------------------------------------------------------------------------------
\newpage
\pagenumbering{arabic} 
%----------------------------------------------------------------------------------------
%
% Introduction
%
%----------------------------------------------------------------------------------------
\section{Introduction}

\subsection{Outline}

\subsection{Notation}

Let $ f: \mathbb{R} \rightarrow \mathbb{R} $ be an arbitrary function. We denote
the element-wise application of $f$ on a vector $x \in \mathbb{R}^n$ with:

\begin{equation*}
	[f]_n: \mathbb{R}^n \rightarrow \mathbb{R}^n, \quad 
	[f]_n(x_1, ... x_n) := \begin{pmatrix}
		f(x_1) \\
		\vdots \\
		f(x_n)
	\end{pmatrix}
\end{equation*}

\todo{Introduce block matrix notation for general functions}

%----------------------------------------------------------------------------------------
%
% Content
%
%----------------------------------------------------------------------------------------
\newpage
\section{Problem setup}

\section{Architecture}

In this section we first recapitulate the symplectic layers as proposed by Jin et. al 
in \cite{Jin2020} \todo{How to cite with name?}. Additionally we supplement proofs for 
symplecticity. A similar architecture was initially proposed by Deco in \cite{Deco1995} 
\todo{Have a deeper look at \cite{Deco1995}}.


\subsection{Linear layers}

\subsection{Activation layers}

\subsection{Gradient layers}

\begin{equation*}
	\mathcal{G} \begin{pmatrix}
		q \\
		p
	\end{pmatrix} := \begin{bmatrix}
		I & 0 \\
		\hat{\sigma} & I
	\end{bmatrix} \begin{pmatrix}
		q \\
		p
	\end{pmatrix} := \begin{pmatrix}
		q \\
		K^T diag(a) [\sigma]_n(Kq + b) + p
	\end{pmatrix}
\end{equation*}

where $K \in \mathbb{R}^{nxd}$ and $a,b \in \mathbb{R}^n$.

\newtheorem*{glayer}{Lemma}
\begin{glayer}
	Given an activation function $\sigma \in C^1$ the gradient layer $\mathcal{G}$ is symplectic.
\end{glayer}
\begin{proof}
	Let $ V(p) := 1_n^Tdiag(a)[\int \sigma]_n(Kp+b)$ \todo{Notation for 1-vector?}

	Then the Jacobi matrix $DV(p)$ is given by \todo{Notation Jacobi matrix}
	\begin{equation*}
		DV(p) = 1_n^Tdiag(a)diag\left([\sigma]_n(Kp+b)\right)K
	\end{equation*}

	Thus
	\begin{align*}
		\nabla V(p) &= \left(DV(p)\right)^T \\
		&= K^Tdiag(a)diag\left([\sigma]_n(Kp+b)\right)1_n \\
		&= K^Tdiag(a)[\sigma]_n(Kp+b)
	\end{align*}

	\todo{Hessian $(D(\nabla V))(p)$} is symmetric because $V \in C^2$. With Lemma \todo{reference Lemma
	which shows that symmetry implies symplecticity} follows symplecticity of $\mathcal{G}$.

	\todo{Rather create separate lemma for general $V: \mathbb{R}^d \rightarrow \mathbb{R}$}
\end{proof}

\todo{Relation to 'Generating functions' in literature?}

%----------------------------------------------------------------------------------------
%
% Resume
%
%----------------------------------------------------------------------------------------
\clearpage\newpage\null %empty page
\newpage
\section{R\'esum\'e}
\subsection{Summary and conclusion}

\subsection{Outlook}

%----------------------------------------------------------------------------------------
%
% Appendices
%
%----------------------------------------------------------------------------------------
\clearpage\newpage\null %empty page
\newpage
\begin{appendices}
\addtocontents{toc}{\protect\setcounter{tocdepth}{1}}
\section{First Appendix Section}

\newpage~\newpage
\section{Declaration of authorship}

\vspace{3cm}

\begin{table}[h!]
\centering
\begin{tabular}{|p{13cm}|}
\hline\\
	I hereby certify
	\begin{enumerate}
		\item that this thesis has been composed by me and is based on my own work, unless stated otherwise,
		\item that all direct or indirect sources used are acknowledged as references and all extracts from work of others, either verbatim or in spirit, are stated as such,
		\item that neither the thesis itself nor parts of this thesis have been part of another examination procedure,
		\item that neither the thesis itself nor parts of this thesis have been published and
		\item that all copies of this thesis, either digital or printed, coincide.
	\end{enumerate}
	Therewith, this declaration of authorship is in accordance with the examination regulations from 29th July 2013 of the master's program \emph{Simulation Technology} of the University of Stuttgart.\\\\
\hline
\end{tabular}
\end{table}

\vspace{4cm}
\hrulefill\\
Name
\hspace{7cm}
Date, City, Signature
\end{appendices}
%----------------------------------------------------------------------------------------
%
% BIBLIOGRAPHY
%
%----------------------------------------------------------------------------------------
\clearpage\newpage\null %empty page
\newpage
\addtocontents{toc}{\protect\setcounter{tocdepth}{1}}
\addcontentsline{toc}{section}{References}
\bibliographystyle{abbrvnat}
\bibliography{../../literature/references.bib}

\end{document}
